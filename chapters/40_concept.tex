%%%%%%%%%%%%%%%%%%%%%%%%%%%%%%%%%%%%%%%%%%%%%%%%%%%%%%%%%%%%%%%%%%%%%%%%%%%%%
\chapter{Konzept}\label{chap:concept}
%%%%%%%%%%%%%%%%%%%%%%%%%%%%%%%%%%%%%%%%%%%%%%%%%%%%%%%%%%%%%%%%%%%%%%%%%%%%%
\chapterstart

Der aktuell implementierte Prozess ist in den Phasen Entwicklung und Systemtest bereits agil konzipiert. 
Der Ablauf der Anforderungsdefinition, Produktspezifikation und dem Entwicklungs-Start sind einem klassischen Wasserfallmodell nachempfunden. Vor allem die Anforderungsdefinition ist die große Hürde, da hier bereits die Anforderungen recht detailreich vertraglich beschlossen werden.

Dies steht im Gegensatz zum agilen Manifest, welches in seinen 4 Punkten das allgemeine Gesamtverhalten beschreibt. 

\textbf{Individuals and interactions} over processes and tools \\
\textbf{Working software} over comprehensive documentation \\
\textbf{Customer collaboration} over contract negotiation \\
\textbf{Responding to change} over following a plan \\ \parencite{Beck:2001}

Um den gesamten Prozess agiler gestalten zu können, muss das agile Konzept über die Entwicklungs-Umsetzung hinaus adaptiert werden. 

\section{Agile Produktspezifikation}

Laut Rubin soll man nicht davon ausgehen, dass man die Pläne zu Beginn bereits richtig definieren kann. Es werden keine Optionen offen gelassen. \parencite[vgl.][S. 248, 249]{Rubin:2012} Die Erstellung des Pflichtenheftes und des Projektplans fordern jedoch genau das. Meilensteine, Zeitpläne und notwendige Aktivitäten für den korrekten Abschluss sind zu diesem Zeitpunkt bereits definiert.

Um dem entgegenzuwirken, sollte die Produktspezifikation Teil des Scrum Prozesses werden. In jedem Sprint trifft sich der Produktverantwortliche mit dem Entwicklungsteam um über die nächsten Anforderungen im Sprint einzulasten. Ein Vorteil dieser Methode ist, da der Produktverantwortliche zu Beginn noch nicht im Klaren über die genau Anforderung ist, dass das sich entwickelnde System mehr Klarheit über die Anforderungen schafft. Außerdem hat das Entwicklungsteam durch den direkten Zugriff zum Produktverantwortlichen die Möglichkeit, die Anforderung besser verstehen zu können. \parencite[vgl.][S. 27]{Ramadan:2016}

Im Fallbeispiel bedeute dies konkret, dass \ac{PD} Teil des Scrum-Teams wird. Es würd kein Pflichtenheft vorab mehr geschrieben werden, sondern das Pflichtenheft würde in jeder Iteration erweitert und detailreicher spezifiziert werden.

Die große Hürde dieser Umstellung ist jedoch die vertragliche Regelung, die bereits in der Anforderungsdefinition beginnt. Aus der Analyse geht hervor, dass mit Abschluss des Pflichtenheftes der Vertrag zustande gekommen ist. Dies ist der Zeitpunkt an dem terminliche Fixpunkte definiert werden. Ebenso die Aufwandsschätzung ist hier abgeschlossen. Um diese Hürde bewältigen zu können, benötigt es einer anderen Form der vertraglichen Regelung. Einen agilen Vertrag.

\section{Agiler Vertrag}

An agile fixed-price contract defines a contractual framework within which time and costs are agreed upon, as well as a structured approach to steer the scope within boundaries and by processes in a defined and controlled manner. Thus, the contractual structure of an agile fixed-price contract reacts to two uncertainties. On the one hand, you never know exactly what details will be needed at the start of a project. On the other hand, you do not always need everything that had originally been considered to be important. \parencite[S. 47]{Opelt:2013}

Für die Umsetzung eines agilen Vertrages, bedarf es eines Änderungswillen im gesamten Unternehmen und den Willen des Kunden agile Methodiken einzusetzen. Wenn dieser vorhanden ist, bedarf es keiner Pflichtenheft Erstellung mehr. Die Anforderungsdefinition kann auf ein Minimum reduziert werden da zu Beginn Anforderungen nur auf einem hohen Abstraktionsniveau definiert werden.

\chapterend