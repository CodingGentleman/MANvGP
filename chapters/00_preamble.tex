%**********************************************************************
% Various Latex packages
%**********************************************************************


% 
\usepackage{ifthen}

% you might need some mathematical expressions:
\usepackage{amsmath}

% with package babel we allow to use language english and german
\usepackage[english, german]{babel}

\usepackage[utf8]{inputenc}	% allows direct input of special chars
\usepackage{setspace}		% permits to set space between lines

% ensure proper appearance of all fonts in pdf:
\usepackage[T1]{fontenc}
%\usepackage{lmodern}		% lmodern after T1 fontenc (_may_ be required)
%\usepackage{times} -- obsolete; use:
\usepackage{mathptmx}		% Times as default text font, maths support
\usepackage{courier}		% provides bold font (required for syntax highlighting in listings)

\usepackage{multirow}		% enables table cells to span multiple rows

\usepackage{parskip}		% paragraphs: no indentation at beginning, but spacing between

\usepackage[pdftex]{graphicx}
\DeclareGraphicsExtensions{.pdf,.jpg,.png}

%**********************************************************************
% Including non-standard packages
%**********************************************************************

\usepackage{acronym}

\usepackage[usenames,dvipsnames,table]{xcolor}	% http://en.wikibooks.org/wiki/LaTeX/Colors
\definecolor{gray20}{gray}{0.8}
\definecolor{gray5}{gray}{0.95}
\definecolor{olivegreen30}{RGB}{155,187,89}	% from table template in MS Word

\usepackage{alltt}

\usepackage{listings}
\lstset{numbers=left, basicstyle=\footnotesize\ttfamily,	% numbers=none if required
	showstringspaces=false,
	captionpos=b, breaklines=false, numbersep=5pt}	% captionpos=b: caption at bottom

% Geometry as defined by FH guidelines:
\usepackage[top=3cm, bottom=3cm, left=3.5cm, right=3cm]{geometry}

\usepackage[super]{nth}     % 1st, 2nd, 3rd,...

%\usepackage{paralist} 		% inline lists
%\usepackage{mdwlist}
\usepackage{enumitem}

\usepackage{float}
\floatstyle{plain}
\restylefloat{figure}
\usepackage{subfig}

\usepackage{textcomp}		% symbols such as \texttimes and \texteuro
%\usepackage{amssymb}		% math. symbols from the American Mathematical Society

\usepackage[Lenny]{fncychap}	% chapter heading styles

\usepackage{csquotes}       % for \enquote, \textquote, \blockquote...

% how to create simple helper commands:
\newcommand{\TODO}[1]
{
{\textcolor{red}{[TODO: #1]}}
}


% how to create more complex new commands:
% BEGIN: chapquote
\newcommand{\chapquote}[2]  % style your own new command
{%
\begin{quote}
\emph{%
``#1''%
}%
\begin{flushright}
{\scriptsize \sffamily [#2]}%
\end{flushright}
\end{quote}
}
% END: chapquote

% Biblatex
% ---------------------------------------------------------------------
\usepackage{csquotes}		% context sensitive quotation; recommended for usage with Biblatex
% Note: date, origdate, eventdate, and urldate require yyyy-mm-dd format
% dd or mm-dd may be omitted
\usepackage[
    backend=biber,
    urldate=long,		% default: short, e.g. 08/15/2010
    style=authoryear-icomp,	% Harvard citation style
    backref,            % if you like (cit. on p. 2)
%   sorting=nty		this is default: sort by name, title, year
%   sortlocale=de_DE	set according to your needs
    natbib=true,		% if you want to use natbib compatible citation commands; do _not_ use package natbib!
    maxbibnames=1000,		% show all authors in the bibliography
]{biblatex}
% Note the default option: ibidpage=true for ibid / ebd

% Strict Harvard style: URL date default is "(Visited on ...)"; so:
% These two BibTeX entries
%	url = {http://...},
%	urldate = {2015-03} -- or 2015, or 2015-03-31
% shall be printed as
%	Available from: <http://...> [March 2015]
\DeclareFieldFormat{urldate}{\mkbibbrackets{#1}}
\DeclareFieldFormat{url}{Available\space from\addcolon\space \url{#1}}
\addbibresource{references.bib}
% ---------------------------------------------------------------------

\usepackage[			% hyperref should be last package loaded
    pdftex,			% driver
    pdftitle={Analyse des Softwareentwicklungsprozess eines Unternehmens},
    pdfsubject={Seminararbeit},
    pdfauthor={Kevin Schmid},
    breaklinks,			% permits line breaks for long links
    bookmarks,			% create Adobe bookmarks
    bookmarksnumbered,		% ... and include section numbers
    linktocpage,		% "make page number, not text, be link on TOC ..."
    colorlinks,			% yes ...
    linkcolor=black,		% normal internal links;
    anchorcolor=black,		% don't make scientific papers too much colourful => "black"
    citecolor=black,
    urlcolor=blue,		% quite common
    pdfstartview={Fit},		% "Fit" fits the page to the window
    pdfpagemode=UseOutlines,	% open bookmarks in Acrobat
    plainpages=false,		% avoids duplicate page number problem
    pdfpagelabels,
  ]{hyperref}

%**********************************************************************
% Layout adjustments
%**********************************************************************

% page layout (header/footer and page numbers)
%\pagestyle{empty}
\pagestyle{headings}
%\pagestyle{fancy}

% settings for structure
\setcounter{secnumdepth}{3}
%\setcounter{tocdepth}{2}
\setcounter{tocdepth}{1}

% footnotes: no indent, hanging
\usepackage[hang,flushmargin]{footmisc}

%**********************************************************************
% LaTeX macros and commands
%**********************************************************************

% new command to start a chapter (no page number)
\newcommand{\chapterstart}{\thispagestyle{empty}}

% new command to close a chapter (flush, i.e. print remaining figures and tables)
\newcommand{\chapterend}{\newpage{\pagestyle{empty}\cleardoublepage}}

% new environment for smaller paragraphs
\newenvironment{spar}
{\begingroup \leftskip 0.7cm \rightskip\leftskip}
{\par \endgroup}
% ^^^ must be set here (or use empty line)



% You might define support for further programming languages
% when using listings
\usepackage{color}
\definecolor{lightgray}{rgb}{.9,.9,.9}
\definecolor{darkgray}{rgb}{.4,.4,.4}
\definecolor{purple}{rgb}{0.65, 0.12, 0.82}
\lstdefinelanguage{JavaScript}{
  keywords={break, case, catch, continue, debugger, default, delete, do, else, false, finally, for, function, if, in, instanceof, new, null, return, switch, this, throw, true, try, typeof, var, void, while, with},
  morecomment=[l]{//},
  morecomment=[s]{/*}{*/},
  morestring=[b]',
  morestring=[b]",
  ndkeywords={class, export, boolean, throw, implements, import, this},
  keywordstyle=\color{blue}\bfseries,
  ndkeywordstyle=\color{darkgray}\bfseries,
  identifierstyle=\color{black},
  commentstyle=\color{purple}\ttfamily,
  stringstyle=\color{red}\ttfamily,
  sensitive=true
}



%**********************************************************************
% Special hyphenation rules
%**********************************************************************
\usepackage[export]{adjustbox}
\hyphenation{JOANNEUM}		% extend to your needs


%**********************************************************************
% Different settings for ITM / SWD / IRM / IMS
%**********************************************************************


% ITM = Internettechnik
% ------------------------
\ifthenelse{\equal{\study}{ITM}}{
  \def \theStudyProgramme {Internettechnik}
  \def \isBachelorThesis {}
}
%\fi

% SWD = Software Design
% ------------------------
\ifthenelse{\equal{\study}{SWD}}{
  \def \theStudyProgramme {Software Design}
  \def \isBachelorThesis {}
}

% IRM = IT-Recht & Management
% -------------------------------
\ifthenelse{\equal{\study}{IRM}}{
  \def \theStudyProgramme {IT-Recht \& Management}
  \def \isMasterThesis {}
}

% IMS = IT & Mobile Security
% ------------------------------
\ifthenelse{\equal{\study}{IMS}}{
  \def \theStudyProgramme {IT \& Mobile Security}
  \def \isMasterThesis {}
}


