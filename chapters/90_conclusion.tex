%%%%%%%%%%%%%%%%%%%%%%%%%%%%%%%%%%%%%%%%%%%%%%%%%%%%%%%%%%%%%%%%%%%%%%%%%%%%%
\chapter{Fazit}\label{chap:conclusion}
%%%%%%%%%%%%%%%%%%%%%%%%%%%%%%%%%%%%%%%%%%%%%%%%%%%%%%%%%%%%%%%%%%%%%%%%%%%%%
\chapterstart

Das Fallbeispiel implementiert agile Methodiken in einem kleinen Bereich im gesamten Entwicklungsprozess. Nur Prozesse, die direkt mit dem Entwicklungsteam in Kontakt kommen, sind agil konzipiert.

Eine Ausweitung agiler Methodiken wäre durchaus möglich. Im ersten, einfachen, Schritt kann die Produktspezifikation in den Scrum-Prozess ohne große Probleme eingeführt werden. Bestehende Abläufe müssten nur gering geändert werden. Der große Unterschied bestünde aus der Nähe des \ac{PD} zum Entwicklungsteam. Die Änderung könnte ohne große Hürden im Unternehmen umgesetzt werden. Somit wäre es sinnvoll, dies in einer Probephase zu erproben.

Die weitaus schwierigere Aufgabe ist es, die Idee eines agilen Vertrages im Unternehmens durchzusetzen. Juristische Expertise ist hier gefragt um rechtliche Sicherheit weiterhin gewährleisten zu können. Eine Einführung könnte aber große Effizienz- und Zufriedenheitssteigerungen mit sich bringen.  

\chapterend
